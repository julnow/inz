\begin{description}[leftmargin=2.2cm,font=\normalfont]
\item[Tytuł pracy:]  Implementacja algorytmów uczenia maszynowego do identyfikacji cząstek w eksperymencie CBM
\end{description}

W ostatnich dekadach obserwuje się rosnącą popularność algorytmów uczenia maszynowego (ML) w licznych dziedzinach biznesu i nauki.
Wiele różnych bibliotek ML zostało stworzonych, pozwalając specjalistom spoza dziedziny informatyki na implementację algorytmów uczenia maszynowego, pozwalają tym samym na np. szybszą analizę dużych, skomplikowanych zbiorów danych, utrzymując przy tym wysoką dokładność.
Z tego powodu ML jest też coraz częściej wybierany przez naukowców zajmujących się fizyką wysokich energii.
Niniejsza praca opisuje możliwość zastosowania algorytmów uczenia maszynowego do identyfikacji cząstek w eksperymencie CBM.\\

Modele uczenia maszynowego do rekonstrukcji Kaonów o krótkim czasie życia (z ang. K-short), jak i identyfikacji trzech grup cząstek (jako odpowiednik metody TOF) zostały zaprezentowane w tej pracy, używając symulowanych danych z modeli Monte Carlo, przepuszczonych przez program GEANT4, w którym symulowany jest układ eksperymentu CBM.
Przygotowanie, trening, jak i ewaluacja modelu została zaprezentowana, podczas gdy możliwość wdrożenia modeli ML do oprogramowania eksperymentu została poddana dyskusji w podsumowaniu.


\vspace{0.025\textheight}
\textit{słowa kluczowe:}\\
\textit{uczenie maszynowe, identyfikacja cząstek, zderzenia ciężkich jonów, CBM, FAIR, GSI}