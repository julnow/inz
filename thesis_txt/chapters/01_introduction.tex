%
% Introduction
%
\pagestyle{fancy}
\thispagestyle{fancy}
Machine learning (ML) is one of the fastest-growing technologies in the last decades. 
Its applications can be found in different fields of research, and its popularity is growing in high-energy physics (HEP) as well, as HEP experiments produce big amounts of data.
The application of ML in this field has been recently appearing in an increasing number of publications, such as ''Generative Models for Fast Calorimeter Simulation: the LHCb case'', in which ML allowed physicists to produce a sufficient amount of simulated data needed by the next HL-LHC experiments using limited computing resources (2019) \cite{lhcb}.
It was also used for $\mathbf{\Lambda}$ Hyperon Reconstruction in the CBM experiment (2021) \cite{shahid} which was a significant inspiration for this thesis. 

There are two main goals of this work: using ML for the reconstruction of the short-lived particles (K-short in this example) and identification of three groups of particles (a counterpart of the TOF method), in the planned CBM experiment in FAIR, Darmstadt. 

In the first two chapters, the physical motivation, and the CBM experiment are described. 
In the next chapter, the machine learning concepts are presented. 
The next chapter presents the traditional methods of the reconstruction and identification of particles.  
The following chapter describes reconstruction of K-short particles using ML, while the next one presents TOF identification using ML. 
The last chapter contains a discussion and summary, as well as a description of possible further developments. 