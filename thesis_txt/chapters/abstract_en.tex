\begin{description}[leftmargin=3.2cm,font=\normalfont]
\item[Title of the thesis:] Implementation of machine learning algorithms for particle identification in the CBM experiment
\end{description}

In the last decades, the increasing popularity of machine learning (ML) algorithms has been observed in many different branches of business and science. 
Multiple ML libraries have been developed, allowing specialists from other fields than computer science easy implementation of these algorithms, which enables e.g., a fast analyse of big, complicated data sets, while maintaining high accuracy. 
For this reason, ML is gaining popularity in the high-energy physics community as well. 
In this thesis, the possibility of applying ML algorithms for the identification of particles in the CBM experiment is discussed.\\

Machine learning models for reconstruction of short-lived Kaons (K-short) and identification of three groups of particles (as a counterpart of the TOF method) have been prepared in this work, using data from Monte Carlo models passed through simulated CBM experiment setup in GEANT4. Model preparation, training, and evaluation have been presented, while the possibility of deploying the ML models in the experiment software has been discussed in the summary.




\vspace{0.025\textheight}
\textit{Keywords:}\\
\textit{machine learning, particle identification, heavy-ion collisions, CBM, FAIR, GSI}