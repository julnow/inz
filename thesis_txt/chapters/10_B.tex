\pagestyle{fancy}
\thispagestyle{fancy}

Framgents of code (\emph{ATreePlainer.cpp}) class used for transforming data in AnalysisTree format into PlainTrees, ready for loading into Pandas Dataframe is presented here.

\lstset{language=C++}
\begin{lstlisting}
#include "ATreePlainer.hpp"
#include "AnalysisTree/TaskManager.hpp"

//Initializes variables associated with each AnalysisTree Branch
void ATreePlainer::Init()
{
  //Initating AnalysisTree Task Manager
  auto* man = AnalysisTree::TaskManager::GetInstance();
  auto* chain = man->GetChain();

  //Variables for each AnalysisTree branch
  tofhits_ = ANALYSISTREE_UTILS_GET<AnalysisTree::
    Detector<AnalysisTree::Hit>*>(chain->GetPointerToBranch("TofHits"));
  simulated_= ANALYSISTREE_UTILS_GET<AnalysisTree::
    Particles*>(chain->GetPointerToBranch("SimParticles"));
  tof2sim_match_ = chain->GetMatchPointers().find(config_->GetMatchName(
    "TofHits", "SimParticles"))->second;
  vtxtracks_ = ANALYSISTREE_UTILS_GET<AnalysisTree::Detector<AnalysisTree::
    Track>*>(chain->GetPointerToBranch("VtxTracks"));
  vtx2tof_match_ = chain->GetMatchPointers().find(config_->GetMatchName(
    "VtxTracks", "TofHits"))->second;

  //Initating output PlainTree
  auto out_config = AnalysisTree::TaskManager::GetInstance()->GetConfig();
  AnalysisTree::BranchConfig out_particles("Complex", AnalysisTree::
    DetType::kParticle);
  out_particles.AddField<float>("mass2");
  out_particles.AddField<int>("q");
  man->AddBranch("Complex", plain_branch_, out_particles);
  InitIndices();
}

//Transformation of AnalysisTree into PlainTree
void ATreePlainer::Exec()
{
  plain_branch_ -> ClearChannels();
  auto out_config = AnalysisTree::TaskManager::GetInstance()->GetConfig();

    //Loop over all reconstructed particles from the TOF detector
   for(auto& input_particle : *tofhits_)
   {
       const auto matched_particle_sim_id = tof2sim_match_->GetMatch(
        input_particle.GetId());
       //Matching with input from MC information about particles
       if (matched_particle_sim_id > 0){
         const auto matched_particle_vtx_id = vtx2tof_match_->GetMatchInverted(
            input_particle.GetId());
         //Matching with data from STS+MVD (VtxTracks branch)
         if (matched_particle_vtx_id > 0){
           //from tof
            auto& output_particle = plain_branch_->AddChannel(
                out_config->GetBranchConfig(plain_branch_->GetId()));
            output_particle.SetField(input_particle.GetField<float>(
                mass2_id_tof_), mass2_id_w1_);
            //from simulated
            auto& matched_particle_sim = simulated_->GetChannel(
                matched_particle_sim_id);
            output_particle.SetMass(matched_particle_sim.GetMass());
            output_particle.SetPid(matched_particle_sim.GetPid());
            //from VtxTracks
            auto& matched_particle_vtx = vtxtracks_->GetChannel(
                matched_particle_vtx_id);
            output_particle.SetMomentum3(matched_particle_vtx.GetMomentum3());
            output_particle.SetField(matched_particle_vtx.GetField<int>(
                q_id_vtx_), q_id_w1_);
            }
          }
      }
}

//Indicies for the output PlainTree
void ATreePlainer::InitIndices()
{
  //Variables from the input file, which are not defined by default
   auto in_branch_tof = config_->GetBranchConfig("TofHits");
   mass2_id_tof_     = in_branch_tof.GetFieldId("mass2");
   auto in_branch_vtx = config_->GetBranchConfig("VtxTracks");
   q_id_vtx_         = in_branch_vtx.GetFieldId("q");
  //Variables in the output file
   auto out_config = AnalysisTree::TaskManager::GetInstance()->GetConfig();
   const auto& out_branch = out_config->GetBranchConfig(
    plain_branch_->GetId());
   mass2_id_w1_           = out_branch.GetFieldId("mass2");
   q_id_w1_               = out_branch.GetFieldId("q");
}



\end{lstlisting}