\begin{thebibliography}{99}
	% Zmieniæ na 99, w przypadku, gdy bibliografia liczy wiêcej ni¿ 9 pozycji.
	% W przypadku du¿ej liczby pozycji, wygodniej korzystaæ z BibTeXa.
	% Zalecany wzór wpisu bibliograficznego dla publikacji z czasopisma: nazwiska i inicja³y imion (przy dwóch imionach bez spacji po kropce!) autorów pisane s¹ kapitalikami. Nastêpnie dwukropek i tytu³ pracy pisany kursyw¹. Nastêpnie kropka i nazwa czasopisma, volume (pogrubiony), rok i numery stron. W przypadku podawania przedzia³ów liczbowych u¿ywa siê pó³pauzy (--) bez spacji, a nie ³¹cznika (-)!
	% Zalecany wzór wpisu dla ksi¹¿ki: nazwiska i inicja³y imion kapitalikami; nastêpnie dwukropek i tytu³ kursyw¹. Po kropce numer wydania (opcjonalnie), nazwa wydawnictwa, miejsce i rok wydania.
	\thispagestyle{fancy}
	
	\bibitem{lhcb}
    V.~Chekalina, E.~Orlova, F.~Ratnikov, D.~Ulyanov, A.~Ustyuzhanin and E.~Zakharov,
    ``Generative Models for Fast Calorimeter Simulation: the LHCb case''
    EPJ Web Conf. \textbf{214} (2019), 02034
    doi:10.1051/epjconf/201921402034
    
    \bibitem{shahid}
    S.~Khan, V.~Klochkov, O.~Lavoryk, O.~Lubynets, A.~I.~Khan, A.~Dubla and I.~Selyuzhenkov,
    ``Machine Learning Application for $\mathbf{\Lambda}$ Hyperon Reconstruction in CBM at FAIR,''
    [arXiv:2109.02435 [physics.ins-det]].
    
	\bibitem{zbroszczyk} Hanna Paulina Zbroszczyk. ''Eksperymentalne aspekty badania korelacji femtoskopowych w zderzeniach relatywistycznych ciezkich jonów''. Oficyna Wydawnicza Politechniki Warszawskiej, 2019, pp. 12, 24, 29.
	
	\bibitem{standard model}D.~Castelvecchi,
    ``What\textquoteright{}s next for physics\textquoteright{} standard model? Muon results throw theories into confusion,''
    Nature \textbf{593} (2021) no.7857, 18-19
    doi:10.1038/d41586-021-01033-8
    \bibitem{wiki}
    wikipedia.org
    
    \bibitem{grebieszkow}
    Grebieszkow K.:"Fizyka zderzeń ciężkich jonów", Lectures at Faculty of Physics of WUT, http://www.if.pw.edu.pl/ kperl/HIP/hip.html
    
    \bibitem{progress report}
    "Compressed Baryonic Matter Experiment at FAIR, Progress Report 2020" Senger, Peter et. al (CBM Collaboration) doi:10.15120/GSI-2021-00421.
    
    \bibitem{nupecc}
    Angela Bracco (2017) ''The NuPECC Long Range Plan 2017: Perspectives in Nuclear Physics'', Nuclear Physics News, 27:3, 3-4, DOI: 10.1080/10619127.2017.1352311
    
    \bibitem{hep map}
    CBM Collaboration, EPJA 53 3 (2017) 60
    T.Galatyuk, NPA982 (2019), update (2021)
    
    \bibitem{alice}
    K.~Aamodt \textit{et al.} [ALICE],
    ``The ALICE experiment at the CERN LHC,''
    JINST \textbf{3} (2008), S08002
    doi:10.1088/1748-0221/3/08/S08002
    
    \bibitem{sis}
    T.~Ablyazimov \textit{et al.} [CBM],
    ``Challenges in QCD matter physics --The scientific programme of the Compressed Baryonic Matter experiment at FAIR,''
    Eur. Phys. J. A \textbf{53} (2017) no.3, 60
    doi:10.1140/epja/i2017-12248-y
    [arXiv:1607.01487 [nucl-ex]].
    
    \bibitem{deconfinement}
    M.~Orsaria, H.~Rodrigues, F.~Weber and G.~A.~Contrera,
    ``Quark deconfinement in high-mass neutron stars,''
    Phys. Rev. C \textbf{89} (2014) no.1, 015806
    doi:10.1103/PhysRevC.89.015806
    [arXiv:1308.1657 [nucl-th]].
    
    \bibitem{fair}
    fair-center.eu, accessed 9.12.2021
    
    \bibitem{cbm-experiment} 
    fair-center.eu/for-users/experiments/cbm.html, accessed 9.12.2021
    
    \bibitem{urqmd}
   S.~A.~Bass, M.~Belkacem, M.~Bleicher, M.~Brandstetter, L.~Bravina, C.~Ernst, L.~Gerland, M.~Hofmann, S.~Hofmann and J.~Konopka, \textit{et al.}
    ``Microscopic models for ultrarelativistic heavy ion collisions,''
    Prog. Part. Nucl. Phys. \textbf{41} (1998), 255-369
    doi:10.1016/S0146-6410(98)00058-1
    [arXiv:nucl-th/9803035 [nucl-th]].
    ].
    
    \bibitem{dcm}
    M.~Baznat, A.~Botvina, G.~Musulmanbekov, V.~Toneev and V.~Zhezher,
    ``Monte-Carlo Generator of Heavy Ion Collisions DCM-SMM,''
    Phys. Part. Nucl. Lett. \textbf{17} (2020) no.3, 303-324
    doi:10.1134/S1547477120030024
    [arXiv:1912.09277 [nucl-th]].
    
    \bibitem{slodkowski}
    Słodkowski M.:"Modelowanie Procesów Jądrowych", Lectures at Faculty of Physics of WUT, http://efizyka.if.pw.edu.pl/MPJ
    
    
    \bibitem{geant4}
    S.~Agostinelli \textit{et al.} [GEANT4],
    ``GEANT4--a simulation toolkit,''
    Nucl. Instrum. Meth. A \textbf{506} (2003), 250-303
    doi:10.1016/S0168-9002(03)01368-8
    
    \bibitem{atree}
    D.~Blau, O.~Golosov, E.~Kashirin, V.~Klochkov and I.~Selyuzhenkov,
    ``Performance studies for collective flow measurements with CBM at FAIR,''
    J. Phys. Conf. Ser. \textbf{1390} (2019) no.1, 012027
    doi:10.1088/1742-6596/1390/1/012027
    
    \bibitem{ibm}
    https://www.ibm.com/cloud/learn/machine-learning, accessed 17.12.2021
    
    \bibitem{ml0}
    A. L. Samuel, "Some Studies in Machine Learning Using the Game of Checkers," in IBM Journal of Research and Development, vol. 3, no. 3, pp. 210-229, July 1959, doi: 10.1147/rd.33.0210.
    
    
    \bibitem{zenia}
    E. Lavrik, et. al., “Optical Inspection of the Silicon Micro-strip Sensors for the CBM Experiment employing Artificial Intelligence''
    arXiv:2107.07714
    
    \bibitem{xgboost}
    Chen, Tianqi, and Carlos Guestrin. "Xgboost: A scalable tree boosting system." Proceedings of the 22nd acm sigkdd international conference on knowledge discovery and data mining. 2016.
    [arXiv:1603.02754]
    
    \bibitem{xgboost1}
    https://towardsdatascience.com/https-medium-com-vishalmorde-xgboost-algorithm-long-she\\-may-rein-edd9f99be63d, accessed 18.12.2021
    
    \bibitem{xgboost-doc}
    https://xgboost.readthedocs.io/en/latest/parameter.html, accessed 18.12.2021
    
    \bibitem{compare hyper}
    http://neupy.com/2016/12/17/hyperparameter\_optimization\_for\_neural\_networks.html, accessed 18.12.2021
    
    \bibitem{ostrowski}
    Ostrowski J., ''Particle Identification Framework extension for multidimensional fitting and mismatch studies'', presentation on the 38th CBM Collaboration Meeting
    
    \bibitem{kdecay1}
    http://hyperphysics.phy-astr.gsu.edu/hbase/Particles/kaon.html, accessed 19.12.2021
    
    \bibitem{kfp}
    Zyzak, PhD thesis, 165 (2016)
    
    \bibitem{lubynets}
    Internal CBM documents: communication with O. Lubynets
    
    \bibitem{get involved}
    Nowak J. ''Optimizing K0S reconstruction at different collision energies using Machine Learning algorithms'', Internship and Training Project Report from the Get\_Inolved programme
    
    \bibitem{plaintree}
    https://github.com/shahidzk1/CBM\_ML\_Lambda\_Library.git
    
    \bibitem{uproot}
    https://github.com/scikit-hep/uproot4
    
    \bibitem{hepnames}
    https://pdg.lbl.gov/2019/reviews/rpp2019-rev-monte-carlo-numbering.pdf, accessed 27.12.2021
    
    \bibitem{cmatrix} https://scikit-learn.org/stable/modules/generated/sklearn.metrics.confusion\_matrix.html, accessed 27.12.2021
    
    \bibitem{treelite}
    https://github.com/dmlc/treelite
    

 


\end{thebibliography}